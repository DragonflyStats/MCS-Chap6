
%% -------------------------------------------------------------------------%
%% - PB on LRTS for LMEs
%% - http://ayeimanol-r.net/2013/11/05/mixed-effects-modeling-four-hour-workshop-part-iv-lmes/

Pinheiro & Bates (2000; p. 88) argue that Likelihood Ratio Test comparisons of models varying in fixed effects tend to be anticonservative i.e. 
will see you observe significant differences in model fit more often than you should. 

I think they are talking, especially, about situations in which the number of model parameter differences (differences between the complex model and 
the nested simpler model) is large relative to the number of observations. 

This is not really a worry for this dataset, but I will come back to the substance of this view, and alternatives to the approach taken here.

