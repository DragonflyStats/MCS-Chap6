

% Failure of Wald CIs
% http://people.upei.ca/hstryhn/stryhn208.pdf

\subsection{Two Options }
\begin{itemize}
\item Wald Type CIs
\item PL Type CIs
\end{itemize}

\subsection{Profile Likelihood Confidence Intervals}
The Profile-likelihood based confidence intervals methods is described in Venzon and Moolgavkar, Journal of the Royal Statistical Society, Series C vol 37, no.1, 1988, pp. 87-94. 

Profile likelihood confidence intervals can be computed for real parameter estimates.

The default confidence intervals for real parameter estimates in the 0-1 interval are based on the standard error and the logit transformation.  
That is, a 95\% confidence interval is computed on the logit estimate, and then these intervals are transformed to the real scale.  


\newpage

\section{Two-tailed testing} A test for equality of variances, based on the likelihood Ratio test, is very simple to implement using existing methodologies. All that is required it to specify the reference model and the relevant nested mode as arguments to the command \texttt{anova()}. The output can be interpreted in the usual way.

\section{One Tailed Testing}
The approach proposed by Roy deals with the question of agreement, and indeed interchangeability, as developed by Bland and Altman's corpus of work. In the view of Dunn, a question relevant to many practitioners is which of the two methods is more precise.

The relationship between precision and the within-item and between-item variability must be established. Roy establishes the equivalence of repeatability and within-item variability, and hence precision.  The method with the smaller within-item variability can be deemed to be the more precise.

\section{Enabling One Tailed Testing}
A useful approach is to compute the confidence intervals for the ratio of within-item standard deviations (equivalent to the ratio of repeatability coefficients), which can be interpreted in the usual manner ( or alternatively, the ratio of the variances). In fact, the ratio of within-item standard deviations, with the attendant confidence interval,  can be determined using a single \texttt{R} command: \texttt{intervals()}.

Pinheiro and Bates (pg 93-95) give a description of how confidence intervals for the variance components are computed. Furthermore a complete set of confidence intervals can be computed to complement the variance component estimates.
However , to facilitate one tailed testing, What is required is the computation of the variance ratios of within-item and between-item standard deviations.

A na�ve approach would be to compute the variance ratios by relevant F distribution quantiles. However, the question arises as to the appropriate degrees of freedom. However, Douglas Bates has stated that an alternative approach is required (i.e. Profile Likelihoods)

\begin{quote}
"The omission of standard errors on variance components is intentional.
The distribution of an estimator of a variance component is highly
skewed and obtaining an estimate of the standard deviation of a skewed
distribution is not very useful.  A much better approach is based on
profiling the objective function." (Douglas Bates May 2012)
\end{quote}


\section{Profile Likelihood}
Normal-based confidence intervals for a parameter of interest are inaccurate when the sampling distribution of the estimate is skewed. The technique known as profile likelihood can produce confidence intervals with better coverage. It may be used when the model includes only the variable of interest or several other variables in addition. Profile-likelihood confidence intervals are particularly useful in nonlinear models.

Profile likelihood confidence intervals are based on the log-likelihood function.  
%For a single parameter, likelihood theory shows that the 2 points 1.92 units down from the maximum of the log-likelihood function provide a $95\%$ confidence interval when there is no extrabinomial variation (i.e. c = 1)..  The value 1.92 is half of the chi-square value of 3.84 with 1 degree of freedom.

%Thus, the same confidence interval can be computed with the deviance by adding 3.84 to the minimum of the deviance function, where the deviance is the log-likelihood multiplied by -2 minus the -2 log likelihood value of the saturated model.

\section{Implementation of PL Confidence Intervals}

The suitable calculation of confidence limits for this variance ratio are to be computed using the profile likelihood approach. The \texttt{R} package \texttt{profilelikelihood} will be assessed for feasibility, particularly the command \texttt{profilelikelihood.lme()}



%http://cran.r-project.org/web/packages/ProfileLikelihood/ProfileLikelihood.pdf

%http://lme4.r-forge.r-project.org/slides/2011-03-16-Amsterdam/3Profiling.pdf

%http://lme4.r-forge.r-project.org/slides/2009-07-21-Seewiesen/4PrecisionD.pdf




%\addcontentsline{toc}{section}{Bibliography}

%\bibliography{transferbib}
\end{document}
