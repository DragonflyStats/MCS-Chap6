\section{Interval in Statistical Theory}
In statistical theory, Confidence Intervals are the most commonly encountered interval estimate. However it is not the only type of interval.


%-------------------------------------------------------------------------------------%
\subsection{Confidence Intervals}
A confidence interval covers a population parameter with a stated confidence, that is, a certain proportion of the time.
Confidence limits are limits within which we expect a given population parameter, such as the mean, to lie. Statistical tolerance limits are limits within which we expect a stated proportion of the population to lie.

%-------------------------------------------------------------------------------------%
\subsection{Tolerance Intervals}
A tolerance interval is a statistical interval within which, with some confidence level, a specified proportion of a sampled population falls.

A two-sided tolerance interval consists of two limits between which a given proportion $\beta$ of the population falls with a given confidence
level $1−\alpha$. A one-sided tolerance interval is similar, but consists of a single upper or lower limit.


Let $\{X_1, X_2, \ldots , X_n  \}$, be a random sample for a population with distribution function F(X ). A (\beta,1−\alpha) two-sided
\beta -content tolerance interval $(T_L T_U ) $ , is defined by
\[ \Pr[F(T_U ) − F(T_L ) \geq \beta ] ≥ 1−\alpha\]


%-------------------------------------------------------------------------------------%
\subsection{Prediction Intervals}
A prediction interval is an interval associated with a random variable yet to be observed, with a specified probability of the random variable lying within the interval. 

A prediction interval is an estimate of an interval in which future observations will fall, with a certain probability, given what has already been observed. Prediction intervals are often used in regression analysis.

A prediction interval has the following interpretation. A 95% prediction interval is one where, if you sample some data, construct an interval from that data, and then sample one new data point, there is a 95% chance that the interval will contain that data point. What's crucial here is that this is 95% of the time you repeat the whole procedure. 
If you have an iterated process, this will have the expected interpretation. 




Stephen B. Vardeman's "What about the other intervals?" (The American Statistician, vol. 46, no. 3, pp. 193–197, 1992). If you really do want a prediction interval instead, a nice overview aimed at statistics educators can be found in Scott Preston's "Teaching Prediction Intervals" (Journal of Statistics Education, vol. 8, no. 3, 2000). An explanation of how to compute tolerance intervals for a normal distribution can be found here, in the NIST's Engineering Statistics Handbook.
